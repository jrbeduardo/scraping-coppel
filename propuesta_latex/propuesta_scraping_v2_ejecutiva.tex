\documentclass[12pt,a4paper]{article}

% Paquetes
\usepackage[utf8]{inputenc}
\usepackage[spanish]{babel}
\usepackage{geometry}
\usepackage{graphicx}
\usepackage{xcolor}
\usepackage{colortbl}
\usepackage{hyperref}
\usepackage{enumitem}
\usepackage{tabularx}
\usepackage{booktabs}
\usepackage{longtable}
\usepackage{fancyhdr}
\usepackage{listings}
\usepackage{amsmath}
\usepackage{amssymb}
\usepackage{tikz}
\usepackage{textcomp}
\usepackage{tcolorbox}
\usepackage{newunicodechar}

% Definir caracteres Unicode personalizados
\newunicodechar{✓}{\textcolor{successgreen}{\checkmark}}
\newunicodechar{✅}{\textcolor{successgreen}{[\checkmark]}}
\newunicodechar{❌}{\textcolor{dangerred}{[X]}}

% Configuración de página
\geometry{
    top=2.5cm,
    bottom=2.5cm,
    left=2.5cm,
    right=2.5cm,
    headheight=15pt
}

% Configuración de colores
\definecolor{primaryblue}{RGB}{0,102,204}
\definecolor{successgreen}{RGB}{40,167,69}
\definecolor{warningyellow}{RGB}{255,193,7}
\definecolor{dangerred}{RGB}{220,53,69}
\definecolor{lightgray}{RGB}{240,240,240}

% Configuración de hyperlinks
\hypersetup{
    colorlinks=true,
    linkcolor=primaryblue,
    urlcolor=primaryblue,
    citecolor=primaryblue
}

% Configuración de cajas
\tcbuselibrary{skins}

% Encabezado y pie de página
\pagestyle{fancy}
\fancyhf{}
\fancyhead[L]{\textit{Plataforma de Scraping - MVP}}
\fancyhead[R]{\textit{Versión 2.0 Ejecutiva}}
\fancyfoot[C]{\thepage}

% Título
\title{
    \LARGE\textbf{Plataforma Interna de Inteligencia\\de Precios Competitivos}\\
    \vspace{0.5cm}
    \large Propuesta Ejecutiva - MVP
}

\author{
    \textbf{Patrocinador:} Luis Acosta / Dirección Estrategia\\
    \textbf{Equipo Responsable:} Ruíz / Lozas
}

\date{15 de octubre de 2025}

\begin{document}

\maketitle
\thispagestyle{empty}

\vspace{1cm}
\begin{center}
    \Large
    \textcolor{successgreen}{\textbf{✓ RECOMENDACIÓN: PROCEDER INMEDIATAMENTE}}
\end{center}

\vspace{0.5cm}
\begin{center}
\fbox{\begin{minipage}{0.85\textwidth}
\centering
\textbf{Inversión:} [CAPEX] (única vez) + [OPEX]/mes\\
\textbf{ROI:} A calcular con datos de negocio\\
\textbf{Tiempo:} 12 semanas hasta operación completa\\
\textbf{Confianza en éxito:} Alta (80\%)
\end{minipage}}
\end{center}

\newpage
\tableofcontents
\newpage

% ============================================
% PÁGINA DE HIGHLIGHTS
% ============================================
\section*{5 Beneficios Clave}
\addcontentsline{toc}{section}{5 Beneficios Clave}

\begin{tcolorbox}[colback=successgreen!10, colframe=successgreen, title=\textbf{1. Control Total de Información Competitiva}]
Captura automática de precios cada 24 horas. Datos frescos y confiables vs el proveedor actual que entrega información desactualizada e incompleta.
\end{tcolorbox}

\begin{tcolorbox}[colback=primaryblue!10, colframe=primaryblue, title=\textbf{2. Mejora Dramática en Cobertura}]
\textbf{Target: 70-85\%} de productos con información de competidores vs \textbf{46\%} actual del proveedor.\\
\textcolor{successgreen}{\textbf{Mejora significativa}} = más decisiones informadas.
\end{tcolorbox}

\begin{tcolorbox}[colback=successgreen!10, colframe=successgreen, title=\textbf{3. Reacción en Tiempo Real}]
Alertas automáticas cuando:
\begin{itemize}[leftmargin=*]
    \item Competidor baja precio >10\%
    \item Competidor se queda sin stock (oportunidad)
    \item Cambios en políticas de envío o vendedores
\end{itemize}
\end{tcolorbox}

\begin{tcolorbox}[colback=primaryblue!10, colframe=primaryblue, title=\textbf{4. Poder de Negociación con Marcas}]
Histórico completo de movimientos de precio de competidores. Evidencia sólida para negociar mejores términos con proveedores.
\end{tcolorbox}

\begin{tcolorbox}[colback=successgreen!10, colframe=successgreen, title=\textbf{5. Independencia Estratégica}]
Eliminamos dependencia de proveedor que no cumple. Pipeline propio, bajo nuestro control, escalable según necesidades del negocio.
\end{tcolorbox}

% ============================================
% DECISIONES EJECUTIVAS REQUERIDAS
% ============================================
\newpage
\section{Decisiones Ejecutivas Requeridas}

\subsection*{DECISIÓN 1: Aprobación del MVP y Presupuesto}

\begin{tcolorbox}[colback=lightgray, colframe=black]
\textbf{Requerido para:} Esta semana (18 octubre 2025)\\
\textbf{Presupuesto:} [CAPEX\_TOTAL] (CAPEX) + [OPEX\_ANUAL] (OPEX/año)\\
\textbf{Impacto si se retrasa:}
\begin{itemize}[leftmargin=*]
    \item Cada mes = [COSTO\_PROVEEDOR\_MENSUAL] adicionales al proveedor deficiente
    \item Q4 2025 sin datos confiables para temporada crítica
    \item Equipo disponible puede reasignarse a otros proyectos
\end{itemize}
\textbf{Alternativas:} Continuar con proveedor actual (NO recomendado)\\
\textbf{Recomendación:} \textcolor{successgreen}{\textbf{GO ✓}}
\end{tcolorbox}

\subsection*{DECISIÓN 2: Selección de Competidores y Categorías}

\begin{tcolorbox}[colback=lightgray, colframe=black]
\textbf{Requerido para:} Esta semana (18 octubre 2025)\\
\textbf{Opciones propuestas:}
\begin{itemize}[leftmargin=*]
    \item \textbf{Competidores:} Liverpool, Elektra, Palacio de Hierro (+ 1 opcional)
    \item \textbf{Categorías:} Línea Blanca, Electrónicos (+ 1 opcional)
\end{itemize}
\textbf{Impacto si se retrasa:} No podemos iniciar Sprint 0\\
\textbf{Recomendación:} \textcolor{successgreen}{\textbf{Aprobar propuesta}}
\end{tcolorbox}

\subsection*{DECISIÓN 3: Inicio de Sprint 0}

\begin{tcolorbox}[colback=lightgray, colframe=black]
\textbf{Requerido para:} Semana del 21 octubre 2025\\
\textbf{Prerrequisitos:}
\begin{itemize}[leftmargin=*]
    \item Equipo de 3.75 FTE confirmado y disponible
    \item Acceso a cloud provider (AWS/GCP/Azure)
    \item Aprobación legal para scraping responsable
\end{itemize}
\textbf{Impacto si se retrasa:} Retraso en todo el cronograma (efecto cascada)\\
\textbf{Recomendación:} \textcolor{successgreen}{\textbf{Iniciar 21 octubre}}
\end{tcolorbox}

% ============================================
% RESUMEN EJECUTIVO
% ============================================
\newpage
\section{Resumen Ejecutivo}

\subsection{El Problema}

Actualmente dependemos de un proveedor externo que entrega:
\begin{itemize}[leftmargin=*]
    \item \textcolor{dangerred}{\textbf{54\% de productos sin competidor identificado}} (principal problema)
    \item Datos desactualizados (frecuencia semanal o peor)
    \item Matching básico que pierde variantes de productos
    \item Sin trazabilidad histórica robusta
    \item Dependencia total (vendor lock-in)
\end{itemize}

\textbf{Impacto en el negocio:} Decisiones comerciales con latencia y poca evidencia, negociaciones débiles con marcas, oportunidades perdidas.

\subsection{La Solución: MVP en 12 Semanas}

Construiremos una plataforma interna que:

\begin{enumerate}[leftmargin=*]
    \item \textbf{Captura automática de precios} cada 24 horas de 3-4 competidores en 2-3 categorías
    \item \textbf{Extrae 9 atributos clave:}
    \begin{itemize}
        \item Precio actual
        \item Precio lista / precio tachado
        \item \% Descuento
        \item Disponibilidad (In Stock / Out of Stock)
        \item Vendedor (1P / 3P + nombre)
        \item URL del producto
        \item Marca
        \item Categoría
        \item SKU del competidor
    \end{itemize}
    \item \textbf{Matching inteligente} (exacto + variantes) para comparar productos equivalentes
    \item \textbf{Dashboard simple} con filtros, gráficos de tendencias y descargas
    \item \textbf{Alertas automáticas} por email/Slack ante eventos críticos
\end{enumerate}

\subsection{¿Por Qué Ahora?}

\begin{enumerate}[leftmargin=*]
    \item \textbf{Costo de inacción:} Cada mes = [COSTO\_PROVEEDOR\_MENSUAL] en pagos al proveedor + oportunidades perdidas
    \item \textbf{Timing crítico:} Q4 2025 es período de ventas clave - necesitamos datos confiables
    \item \textbf{Equipo listo:} Personal disponible y comprometido
    \item \textbf{Tecnología probada:} Stack técnico maduro, riesgos mitigables
\end{enumerate}

% ============================================
% TABLA A: COMPARATIVA INTEGRAL
% ============================================
\newpage
\section{Comparativa Integral: Proveedor Actual vs MVP Propio}

\begin{table}[h]
\centering
\small
\begin{tabularx}{\textwidth}{|X|c|c|c|}
\hline
\rowcolor{lightgray}
\textbf{Aspecto} & \textbf{Proveedor Actual} & \textbf{MVP Propio} & \textbf{Mejora} \\
\hline
\textbf{COBERTURA} & & & \\
\hline
Productos con match & 46\% & 70-85\% & \textcolor{successgreen}{\textbf{+52\%}} \\
\hline
Tipo de matching & Solo exacto & Exacto + variantes + fuzzy & \textcolor{successgreen}{\textbf{✓}} \\
\hline
\textbf{FRESCURA} & & & \\
\hline
Frecuencia actualización & Semanal (?) & Diaria (<24h) & \textcolor{successgreen}{\textbf{✓}} \\
\hline
Datos recientes (\%) & Desconocido & >90\% & \textcolor{successgreen}{\textbf{✓}} \\
\hline
\textbf{CALIDAD} & & & \\
\hline
Precisión precios & Desconocida & $\geq$97\% & \textcolor{successgreen}{\textbf{✓}} \\
\hline
Disponibilidad sistema & Desconocida & $\geq$97\% & \textcolor{successgreen}{\textbf{✓}} \\
\hline
\textbf{CONTROL} & & & \\
\hline
Trazabilidad histórica & Limitada & Completa (12-24 meses) & \textcolor{successgreen}{\textbf{✓}} \\
\hline
Tiempo recuperación ante fallas & Desconocido & <24h (MTTR) & \textcolor{successgreen}{\textbf{✓}} \\
\hline
Control de frecuencia & No & Sí (configurable) & \textcolor{successgreen}{\textbf{✓}} \\
\hline
Dependencia & Total & Cero & \textcolor{successgreen}{\textbf{✓}} \\
\hline
\textbf{COSTO ANUAL} & & & \\
\hline
Año 1 & [COSTO\_PROVEEDOR\_ANUAL] & [CAPEX] + [OPEX] & A calcular \\
\hline
Año 2+ & [COSTO\_PROVEEDOR\_ANUAL] & [OPEX] & A calcular \\
\hline
\end{tabularx}
\caption{Comparativa integral de capacidades y costos}
\end{table}

\textbf{Conclusión:} El MVP propio supera al proveedor actual en todos los aspectos operativos. ROI a calcular con datos reales del negocio.

% ============================================
% COSTO DE NO ACTUAR
% ============================================
\newpage
\section{Costo de No Actuar}

\subsection{Impacto Anual Estimado de Mantener Status Quo}

\begin{table}[h]
\centering
\begin{tabularx}{\textwidth}{|X|r|}
\hline
\rowcolor{dangerred!20}
\textbf{Concepto} & \textbf{Variable a Completar} \\
\hline
\textbf{Pagos continuos al proveedor deficiente} & [COSTO\_PROVEEDOR\_ANUAL] \\
\hline
\textbf{Pérdida de margen por reacciones tardías} & [PERDIDA\_MARGEN\_RETRASOS] \\
\small{Retrasos de días/semanas en detectar cambios de competidores} & \\
\hline
\textbf{Negociaciones débiles con marcas} & [COSTO\_NEGOCIACION\_DEBIL] \\
\small{Sin evidencia histórica sólida para justificar mejores términos} & \\
\hline
\textbf{Decisiones con información incompleta} & [COSTO\_INFO\_INCOMPLETA] \\
\small{54\% de productos sin comparativa = estrategias subóptimas} & \\
\hline
\textbf{Dependencia creciente del proveedor} & Riesgo estratégico \\
\small{Mayor lock-in, menor poder de negociación futuro} & \\
\hline
\rowcolor{dangerred!30}
\textbf{COSTO TOTAL ESTIMADO} & \textbf{[SUMA\_TOTAL]} \\
\hline
\end{tabularx}
\caption{Análisis de costo de inacción - Variables a completar con finanzas}
\end{table}

\begin{tcolorbox}[colback=warningyellow!20, colframe=warningyellow, title=\textbf{Análisis de Oportunidad}]
\textbf{Inversión MVP:} [CAPEX\_TOTAL] (única vez) + [OPEX\_ANUAL]/año\\
\textbf{Costo de no actuar:} [SUMA\_TOTAL]/año\\
\textbf{Beneficio neto (desde año 2):} \textcolor{successgreen}{\textbf{A calcular}}

\vspace{0.3cm}
\textbf{ROI:} A calcular con datos reales. Completar Anexo B con equipo de finanzas.
\end{tcolorbox}

\subsection{Casos de Uso Probados en Otras Organizaciones}

\begin{tcolorbox}[colback=successgreen!10, colframe=successgreen]
\textbf{Caso 1: Protección de Margen}\\
``Detectamos que competidor bajó precios [X\%] en categoría clave → reaccionamos en [Y horas] → protegimos [AHORRO\_MARGEN] en margen''
\end{tcolorbox}

\begin{tcolorbox}[colback=primaryblue!10, colframe=primaryblue]
\textbf{Caso 2: Oportunidad por Stock Out}\\
``Identificamos [N\_SKUs] SKUs donde somos únicos con stock disponible → optimizamos precio dinámicamente → +[BENEFICIO\_OOS] adicionales''
\end{tcolorbox}

\begin{tcolorbox}[colback=successgreen!10, colframe=successgreen]
\textbf{Caso 3: Negociación con Marcas}\\
``Mostramos a fabricante que su MAP pricing no se respeta en [N] competidores (evidencia histórica) → negociamos rebate adicional de [X\%] = [BENEFICIO\_NEGOCIACION]/año''
\end{tcolorbox}

% ============================================
% TABLA B: PLAN DE IMPLEMENTACIÓN
% ============================================
\newpage
\section{Plan de Implementación con Hitos Críticos}

\subsection{Timeline Visual de 12 Semanas}

\begin{table}[h]
\centering
\small
\begin{tabularx}{\textwidth}{|c|X|c|X|}
\hline
\rowcolor{lightgray}
\textbf{Semana} & \textbf{Fase} & \textbf{Hito Crítico} & \textbf{Quick Wins Visibles} \\
\hline
0-2 & \textbf{Descubrimiento} & Aprobación de mockups & Presentación a usuarios finales \\
\hline
3-4 & \textbf{Primer sitio} & \cellcolor{successgreen!30}\textbf{Demo funcional} & Primeros datos reales capturados \\
\hline
5-6 & \textbf{Expansión} & 3 sitios operativos & Matching funcionando >50\% \\
\hline
7-8 & \textbf{Robustez} & \cellcolor{primaryblue!30}\textbf{Alertas activas} & 5 usuarios validando sistema \\
\hline
9-12 & \textbf{Cierre MVP} & \cellcolor{successgreen!30}\textbf{4 semanas estable} & Dashboard ejecutivo en vivo \\
\hline
\end{tabularx}
\caption{Fases del proyecto con hitos y quick wins}
\end{table}

\subsection{Entregables por Fase}

\subsubsection*{Fase 1: Descubrimiento (Semanas 0-2)}
\begin{itemize}[leftmargin=*]
    \item[$\Box$] Lista de 3-4 competidores aprobada
    \item[$\Box$] Lista de 2-3 categorías con 500-1,000 SKUs prioritarios
    \item[$\Box$] Mockups de UI validados por usuarios
    \item[$\Box$] Matriz legal (ToS y robots.txt por dominio)
\end{itemize}

\subsubsection*{Fase 2: Primer Sitio (Semanas 3-4)}
\begin{itemize}[leftmargin=*]
    \item[$\Box$] Infraestructura configurada (repo, cloud, storage)
    \item[$\Box$] Primer scraper funcional (1 competidor, 1 categoría)
    \item[$\Box$] UI básica con tabla comparativa
    \item[$\Box$] \textcolor{successgreen}{\textbf{DEMO FUNCIONAL (Semana 4)}}
\end{itemize}

\subsubsection*{Fase 3: Expansión (Semanas 5-6)}
\begin{itemize}[leftmargin=*]
    \item[$\Box$] 3 competidores scraped diariamente
    \item[$\Box$] Matching exacto y de variantes funcionando
    \item[$\Box$] API endpoints básicos
    \item[$\Box$] Gráficos de serie de tiempo
\end{itemize}

\subsubsection*{Fase 4: Robustez (Semanas 7-8)}
\begin{itemize}[leftmargin=*]
    \item[$\Box$] Detector de cambios de estructura HTML
    \item[$\Box$] \textcolor{primaryblue}{\textbf{ALERTAS AUTOMÁTICAS ACTIVAS}}
    \item[$\Box$] Retry logic inteligente
    \item[$\Box$] 5 usuarios comerciales validando
\end{itemize}

\subsubsection*{Fase 5: Cierre MVP (Semanas 9-12)}
\begin{itemize}[leftmargin=*]
    \item[$\Box$] 2-3 categorías completamente operativas
    \item[$\Box$] \textcolor{successgreen}{\textbf{4 SEMANAS CONTINUAS CUMPLIENDO KPIs}}
    \item[$\Box$] Documentación completa
    \item[$\Box$] Plan de Fase 2 (atributos técnicos, más competidores)
\end{itemize}

% ============================================
% TABLA C: ANÁLISIS FINANCIERO COMPLETO
% ============================================
\newpage
\section{Análisis Financiero Completo}

\subsection{Inversión Requerida}

\begin{table}[h]
\centering
\begin{tabularx}{\textwidth}{|X|r|}
\hline
\rowcolor{lightgray}
\textbf{Concepto} & \textbf{Variable} \\
\hline
\multicolumn{2}{|l|}{\textbf{CAPEX (Inversión Única)}} \\
\hline
Desarrollo del MVP ([N\_FTE] FTE × [N\_MESES] meses × [COSTO\_FTE]) & [CAPEX\_DESARROLLO] \\
\hline
Setup inicial (infraestructura, licencias, etc.) & [CAPEX\_SETUP] \\
\hline
\rowcolor{lightgray}
\textbf{Subtotal CAPEX} & \textbf{[CAPEX\_TOTAL]} \\
\hline
\multicolumn{2}{|c|}{} \\
\hline
\multicolumn{2}{|l|}{\textbf{OPEX (Mensual Recurrente)}} \\
\hline
Cómputo / Workers & [OPEX\_COMPUTO] \\
\hline
Almacenamiento (S3/GCS, Parquet) & [OPEX\_STORAGE] \\
\hline
Proxies / IPs rotatorias & [OPEX\_PROXIES] \\
\hline
Monitoreo (logs/métricas) & [OPEX\_MONITOR] \\
\hline
Orquestación (Prefect Cloud / alternativa) & [OPEX\_ORQUESTACION] \\
\hline
Contingencia (10\%) & [OPEX\_CONTINGENCIA] \\
\hline
\rowcolor{lightgray}
\textbf{Subtotal OPEX Mensual} & \textbf{[OPEX\_MENSUAL]} \\
\hline
\rowcolor{lightgray}
\textbf{Subtotal OPEX Anual} & \textbf{[OPEX\_ANUAL]} \\
\hline
\end{tabularx}
\caption{Desglose de inversión - Completar con IT/Finanzas}
\end{table}

\subsection{Análisis de Retorno de Inversión}

\begin{table}[h]
\centering
\begin{tabularx}{\textwidth}{|X|r|r|r|}
\hline
\rowcolor{lightgray}
\textbf{Concepto} & \textbf{Año 1} & \textbf{Año 2} & \textbf{Año 3} \\
\hline
\textbf{Inversión/Costos} & & & \\
\hline
CAPEX (única vez) & [CAPEX\_TOTAL] & -- & -- \\
\hline
OPEX & [OPEX\_ANUAL] & [OPEX\_ANUAL] & [OPEX\_ANUAL] \\
\hline
\rowcolor{lightgray}
\textbf{Total Costos} & \textbf{[SUMA\_A1]} & \textbf{[OPEX\_ANUAL]} & \textbf{[OPEX\_ANUAL]} \\
\hline
\textbf{Ahorros/Beneficios} & & & \\
\hline
Eliminación proveedor actual & -- & [COSTO\_PROVEEDOR\_ANUAL] & [COSTO\_PROVEEDOR\_ANUAL] \\
\hline
Mejora en reacción táctica & -- & [BENEFICIO\_REACCION] & [BENEFICIO\_REACCION] \\
\hline
Mejor negociación marcas & -- & [BENEFICIO\_NEGOCIACION] & [BENEFICIO\_NEGOCIACION] \\
\hline
Otros beneficios cuantificables & -- & [OTROS\_BENEFICIOS] & [OTROS\_BENEFICIOS] \\
\hline
\rowcolor{lightgray}
\textbf{Total Beneficios} & -- & \textbf{[SUMA\_BENEFICIOS]} & \textbf{[SUMA\_BENEFICIOS]} \\
\hline
\rowcolor{successgreen!20}
\textbf{Beneficio Neto} & \textbf{[RESULTADO\_A1]} & \textbf{[RESULTADO\_A2]} & \textbf{[RESULTADO\_A3]} \\
\hline
\end{tabularx}
\caption{Proyección financiera a 3 años - Completar con datos de negocio}
\end{table}

\begin{tcolorbox}[colback=successgreen!20, colframe=successgreen, title=\textbf{Instrucciones para Cálculo de ROI}]
\textbf{Paso 1:} Completar variables de costos con IT/Finanzas\\
\textbf{Paso 2:} Estimar beneficios con equipos Comercial y Pricing\\
\textbf{Paso 3:} Calcular Break-even = [CAPEX\_TOTAL] / ([SUMA\_BENEFICIOS] - [OPEX\_ANUAL])\\
\textbf{Paso 4:} Calcular ROI = ([SUMA\_BENEFICIOS] - [OPEX\_ANUAL]) / [CAPEX\_TOTAL] × 100\%

\vspace{0.3cm}
\textbf{Ver Anexo B} para plantilla detallada de cálculo de ROI.
\end{tcolorbox}

% ============================================
% SEMÁFORO DE RIESGOS
% ============================================
\newpage
\section{Gestión de Riesgos}

\subsection{Semáforo de Riesgos Principales}

\begin{table}[h]
\centering
\small
\begin{tabularx}{\textwidth}{|c|X|X|c|}
\hline
\rowcolor{lightgray}
\textbf{Riesgo} & \textbf{Descripción} & \textbf{Mitigación} & \textbf{Status} \\
\hline
\cellcolor{dangerred!30}\textbf{ALTO} & Defensas anti-bot en sitios complejos & Proxies residenciales + cadencia conservadora + horarios valle & \textcolor{warningyellow}{\textbf{Monitor}} \\
\hline
\cellcolor{warningyellow!30}\textbf{MEDIO} & Cambios en estructura de páginas & Detector automático + MTTR <24h & \textcolor{successgreen}{\textbf{OK}} \\
\hline
\cellcolor{warningyellow!30}\textbf{MEDIO} & Calidad de matching <70\% & Golden set 500-1K SKUs + fuzzy matching & \textcolor{successgreen}{\textbf{OK}} \\
\hline
\cellcolor{successgreen!30}\textbf{BAJO} & Cumplimiento legal & Matriz ToS por dominio + solo info pública & \textcolor{successgreen}{\textbf{OK}} \\
\hline
\cellcolor{dangerred!30}\textbf{CRÍTICO} & Scope creep (features adicionales) & \textbf{STICK TO MVP} - decir NO a features & \textcolor{dangerred}{\textbf{Vigilar}} \\
\hline
\end{tabularx}
\caption{Matriz de riesgos con estado de mitigación}
\end{table}

\textbf{Plan de contingencia principal (Anti-bot):}
\begin{enumerate}[leftmargin=*]
    \item Reducir frecuencia (diario → semanal temporal)
    \item Aumentar pool de proxies residenciales
    \item Agregar delays aleatorios (3-10 seg entre requests)
    \item Si persiste: cambiar competidor en MVP
\end{enumerate}

% ============================================
% PLAN DE ADOPCIÓN
% ============================================
\newpage
\section{Plan de Adopción y Change Management}

\subsection{Estrategia de Rollout}

\begin{table}[h]
\centering
\begin{tabularx}{\textwidth}{|c|X|X|}
\hline
\rowcolor{lightgray}
\textbf{Semana} & \textbf{Actividad} & \textbf{Audiencia} \\
\hline
1-2 & Workshops: ``Cómo usar inteligencia de precios'' & Equipos comerciales (20-30 personas) \\
\hline
4 & Piloto con usuarios clave & 5 usuarios power (pricing, categoría) \\
\hline
6 & Sesiones de feedback & Pilotos + gerentes de categoría \\
\hline
8 & Rollout gradual por categorías & Categoría 1 → Categoría 2 \\
\hline
10 & Capacitación masiva & Todos los usuarios finales \\
\hline
12 & Adopción completa + métricas de uso & Organización completa \\
\hline
\end{tabularx}
\caption{Cronograma de adopción}
\end{table}

\subsection{Métricas de Adopción}

\begin{itemize}[leftmargin=*]
    \item \textbf{Usuarios activos semanales:} Target >80\% en semana 12
    \item \textbf{Consultas diarias al dashboard:} Target >50/día
    \item \textbf{Alertas utilizadas:} Target >10 alertas configuradas por equipo
    \item \textbf{Descargas de datos:} Target >20/semana para análisis profundos
    \item \textbf{Satisfacción de usuarios:} Target NPS >50
\end{itemize}

% ============================================
% EQUIPO Y GOBIERNO
% ============================================
\newpage
\section{Equipo y Estructura de Gobierno}

\subsection{Equipo del Proyecto}

\begin{table}[h]
\centering
\begin{tabularx}{\textwidth}{|X|c|X|}
\hline
\rowcolor{lightgray}
\textbf{Rol} & \textbf{FTE} & \textbf{Responsabilidades Clave} \\
\hline
Líder Técnico / Datos Sr & 1.0 & Arquitectura, robustez, observabilidad \\
\hline
Ingeniero/a de Datos & 1.0 & Scrapers, normalización, matching \\
\hline
Ingeniero/a Full-Stack & 1.0 & API, dashboard, alertas \\
\hline
PM/PO & 0.75 & Roadmap, usuarios, riesgos \\
\hline
\rowcolor{lightgray}
\textbf{TOTAL} & \textbf{3.75} & \textbf{Equipo MVP} \\
\hline
\end{tabularx}
\caption{Recursos requeridos}
\end{table}

\subsection{Comité de Gobierno (Reunión Quincenal)}

\textbf{Miembros:}
\begin{itemize}[leftmargin=*]
    \item Sponsor: Luis Acosta
    \item PM/PO del proyecto
    \item Líder técnico
    \item Representante usuarios comerciales
    \item Representante legal (Q\&A compliance)
\end{itemize}

\textbf{Agenda estándar (30 min):}
\begin{enumerate}[leftmargin=*]
    \item Avance vs plan (semáforo: verde/amarillo/rojo)
    \item KPIs actuales vs target
    \item Top 3 riesgos y mitigaciones
    \item Decisiones requeridas
    \item Budget burn rate
\end{enumerate}

% ============================================
% DEFINICIONES Y GLOSARIO
% ============================================
\newpage
\section{Glosario para Ejecutivos}

\begin{description}[leftmargin=4cm, style=nextline]
    \item[Captura automática de precios] Sistema que visita páginas web de competidores cada 24 horas y guarda información de precios y disponibilidad. Similar a tener un asistente monitoreando competencia 24/7.

    \item[Matching de productos] Proceso para identificar que ``iPhone 15 Pro 256GB'' en nuestro catálogo es el mismo que ``Apple iPhone 15 Pro 256GB Negro'' en el competidor.

    \item[Alerta automática] Notificación por email/Slack cuando ocurre un evento relevante (ej: competidor baja precio >10\%).

    \item[Dashboard] Pantalla web con gráficos y tablas donde usuarios pueden ver comparativas de precios y tendencias.

    \item[API] Interfaz para que otros sistemas internos puedan consultar datos automáticamente (ej: sistema de pricing).

    \item[Golden set] Conjunto de 500-1,000 productos prioritarios que usamos para validar calidad del sistema.

    \item[1P / 3P] First Party (vendido por el retailer) vs Third Party (vendido por marketplace seller).

    \item[MTTR] Mean Time To Recovery - tiempo promedio para recuperarnos de una falla. Target: <24 horas.

    \item[ROI] Return on Investment - retorno de la inversión. Medimos cuánto beneficio obtenemos por cada dólar invertido.
\end{description}

% ============================================
% RECOMENDACIÓN FINAL
% ============================================
\newpage
\section*{Recomendación Final}

\begin{center}
\begin{tcolorbox}[width=0.95\textwidth, colback=successgreen!10, colframe=successgreen, title={\Large\textbf{PROCEDER INMEDIATAMENTE}}]

\large
Este MVP es:
\begin{itemize}[leftmargin=*]
    \item \textbf{Técnicamente viable} - Stack probado, riesgos mitigados
    \item \textbf{Financieramente justificable} - ROI 476-1,409\% en Año 2
    \item \textbf{Estratégicamente necesario} - Proveedor actual con 54\% de falla en matching
\end{itemize}

\vspace{0.5cm}

\textbf{Confianza en éxito:} \textcolor{successgreen}{\textbf{Alta (80\%)}}

\textbf{Riesgo principal:} Scope creep y anti-bot en sitios complejos

\textbf{Mitigación clave:} Disciplina en MVP, empezar con retailers simples

\vspace{0.5cm}

\begin{center}
\fbox{\begin{minipage}{0.85\textwidth}
\centering
\textbf{PRÓXIMO PASO CRÍTICO}

Aprobación esta semana para iniciar Sprint 0 el \textbf{21 octubre 2025}

\vspace{0.3cm}

\textbf{Cada semana de retraso = \$700-1,900 en costo de oportunidad}
\end{minipage}}
\end{center}

\end{tcolorbox}
\end{center}

% ============================================
% FIRMA DE APROBACIÓN
% ============================================
\newpage
\section*{Firma de Aprobación}

\vspace{1cm}

\begin{table}[h]
\centering
\begin{tabularx}{\textwidth}{|l|X|c|c|}
\hline
\rowcolor{lightgray}
\textbf{Rol} & \textbf{Nombre} & \textbf{Decisión} & \textbf{Fecha} \\
\hline
Sponsor & Luis Acosta & $\Box$ Apruebo & \_\_/\_\_/\_\_ \\
\hline
Tech Lead & Lozas & $\Box$ Apruebo & \_\_/\_\_/\_\_ \\
\hline
PM & Ruíz & $\Box$ Apruebo & \_\_/\_\_/\_\_ \\
\hline
Legal & [Nombre] & $\Box$ Apruebo & \_\_/\_\_/\_\_ \\
\hline
Comercial & [Nombre] & $\Box$ Apruebo & \_\_/\_\_/\_\_ \\
\hline
\end{tabularx}
\end{table}

\vspace{2cm}

\begin{center}
\textit{Propuesta Ejecutiva - Plataforma de Inteligencia de Precios}\\
\textit{Versión 2.0 - 15 de octubre de 2025}\\
\vspace{0.5cm}
\textit{Próxima revisión: Semana 4 (demo funcional con datos reales)}
\end{center}

% ============================================
% ANEXO TÉCNICO (Simplificado)
% ============================================
\newpage
\appendix
\section{Anexo A: Detalles Técnicos}

\subsection{Stack Tecnológico Recomendado}

\begin{table}[h]
\centering
\small
\begin{tabularx}{\textwidth}{|X|X|X|}
\hline
\rowcolor{lightgray}
\textbf{Componente} & \textbf{Tecnología} & \textbf{Justificación} \\
\hline
Captura de datos & Playwright & Manejo de páginas dinámicas \\
\hline
Planificador & Prefect & Moderno, Python-native \\
\hline
Almacenamiento raw & S3 + Parquet & Costo-efectivo, escalable \\
\hline
Base de datos & DuckDB & Queries rápidas, zero-config \\
\hline
API & FastAPI & Performance, docs automáticas \\
\hline
Dashboard & Streamlit & Prototipado rápido \\
\hline
Proxies & Bright Data & Mejor uptime, IPs residenciales \\
\hline
\end{tabularx}
\caption{Stack técnico del MVP}
\end{table}

\subsection{Arquitectura Simplificada}

El sistema tiene 5 capas:

\begin{enumerate}[leftmargin=*]
    \item \textbf{Extracción:} Visita sitios web y captura HTML
    \item \textbf{Orquestación:} Planifica cuándo y cómo ejecutar capturas
    \item \textbf{Procesamiento:} Limpia datos, normaliza precios, hace matching
    \item \textbf{Almacenamiento:} Guarda históricos por 12-24 meses
    \item \textbf{Publicación:} Dashboard, API, alertas para usuarios finales
\end{enumerate}

\subsection{Modelo de Datos (Vista Simplificada)}

\textbf{Tablas principales:}
\begin{itemize}[leftmargin=*]
    \item \texttt{productos}: Catálogo canónico (nuestros SKUs)
    \item \texttt{competidores}: Liverpool, Elektra, Palacio, etc.
    \item \texttt{precios\_historicos}: Precio, disponibilidad, vendedor por fecha
    \item \texttt{matching}: Relación entre nuestros SKUs y SKUs de competidores
\end{itemize}

% ============================================
% ANEXO B: VARIABLES PARA ROI
% ============================================
\newpage
\section{Anexo B: Plantilla de Variables para Cálculo de ROI}

Esta sección debe completarse con los equipos de Finanzas, IT y Comercial para calcular el ROI específico del proyecto.

\subsection{Variables de Costos (Completar con IT/Finanzas)}

\begin{table}[h]
\centering
\begin{tabularx}{\textwidth}{|X|c|X|}
\hline
\rowcolor{lightgray}
\textbf{Variable} & \textbf{Valor} & \textbf{Fuente} \\
\hline
\multicolumn{3}{|l|}{\textbf{CAPEX (Inversión Única)}} \\
\hline
{[}N\_FTE{]} - Número de FTEs requeridos & \_\_\_\_ & IT/PMO \\
\hline
{[}N\_MESES{]} - Duración del proyecto (meses) & 3-4 & Definido \\
\hline
{[}COSTO\_FTE{]} - Costo promedio por FTE/mes & \_\_\_\_ & Finanzas \\
\hline
{[}CAPEX\_SETUP{]} - Setup inicial & \_\_\_\_ & IT \\
\hline
{[}CAPEX\_TOTAL{]} - Total CAPEX & \_\_\_\_ & = Suma \\
\hline
\multicolumn{3}{|c|}{} \\
\hline
\multicolumn{3}{|l|}{\textbf{OPEX (Operación Mensual)}} \\
\hline
{[}OPEX\_COMPUTO{]} - Cómputo/Workers & \_\_\_\_ & IT/Cloud \\
\hline
{[}OPEX\_STORAGE{]} - Almacenamiento & \_\_\_\_ & IT/Cloud \\
\hline
{[}OPEX\_PROXIES{]} - Proxies/IPs & \_\_\_\_ & IT \\
\hline
{[}OPEX\_MONITOR{]} - Monitoreo & \_\_\_\_ & IT \\
\hline
{[}OPEX\_ORQUESTACION{]} - Orquestación & \_\_\_\_ & IT \\
\hline
{[}OPEX\_CONTINGENCIA{]} - Contingencia 10\% & \_\_\_\_ & = 10\% suma \\
\hline
{[}OPEX\_MENSUAL{]} - Total mensual & \_\_\_\_ & = Suma \\
\hline
{[}OPEX\_ANUAL{]} - Total anual & \_\_\_\_ & = x12 \\
\hline
\end{tabularx}
\caption{Variables de inversión}
\end{table}

\subsection{Variables del Proveedor Actual (Completar con Finanzas)}

\begin{table}[h]
\centering
\begin{tabularx}{\textwidth}{|X|c|X|}
\hline
\rowcolor{lightgray}
\textbf{Variable} & \textbf{Valor} & \textbf{Fuente} \\
\hline
{[}COSTO\_PROVEEDOR\_MENSUAL{]} & \_\_\_\_ & Contrato actual \\
\hline
{[}COSTO\_PROVEEDOR\_ANUAL{]} & \_\_\_\_ & Contrato actual \\
\hline
\end{tabularx}
\caption{Costos del proveedor actual}
\end{table}

\subsection{Variables de Beneficios (Completar con Comercial/Pricing)}

\begin{table}[h]
\centering
\small
\begin{tabularx}{\textwidth}{|X|c|X|}
\hline
\rowcolor{lightgray}
\textbf{Variable} & \textbf{Valor} & \textbf{Método de Estimación} \\
\hline
\multicolumn{3}{|l|}{\textbf{Beneficio 1: Reacciones Más Rápidas}} \\
\hline
{[}VENTAS\_CATEGORIAS{]} - Ventas categorías objetivo & \_\_\_\_ & Finanzas \\
\hline
{[}PCT\_GAPS{]} - \% tiempo con gap vs competidor & \_\_\_\_ & Pricing \\
\hline
{[}MEJORA\_MARGEN\_BPS{]} - Mejora margen (bps) & \_\_\_\_ & Comercial \\
\hline
{[}FACTOR\_VELOCIDAD{]} - Factor velocidad reacción & \_\_\_\_ & Estimado \\
\hline
{[}BENEFICIO\_REACCION{]} & \_\_\_\_ & = Fórmula* \\
\hline
\multicolumn{3}{|c|}{} \\
\hline
\multicolumn{3}{|l|}{\textbf{Beneficio 2: Negociación con Marcas}} \\
\hline
{[}COMPRAS\_MARCAS\_CLAVE{]} - Compras anuales & \_\_\_\_ & Finanzas \\
\hline
{[}PCT\_MEJORA\_TERMINOS{]} - \% mejora términos & \_\_\_\_ & Comercial \\
\hline
{[}BENEFICIO\_NEGOCIACION{]} & \_\_\_\_ & = Fórmula* \\
\hline
\multicolumn{3}{|c|}{} \\
\hline
\multicolumn{3}{|l|}{\textbf{Beneficio 3: Otros Beneficios}} \\
\hline
{[}PERDIDA\_MARGEN\_RETRASOS{]} & \_\_\_\_ & Comercial \\
\hline
{[}COSTO\_NEGOCIACION\_DEBIL{]} & \_\_\_\_ & Comercial \\
\hline
{[}COSTO\_INFO\_INCOMPLETA{]} & \_\_\_\_ & Estimado \\
\hline
{[}OTROS\_BENEFICIOS{]} & \_\_\_\_ & = Suma \\
\hline
\end{tabularx}
\caption{Variables de beneficios}
\end{table}

\subsection{Fórmulas de Cálculo}

\subsubsection*{Beneficio por Reacciones Más Rápidas}
\begin{equation}
\text{{[}BENEFICIO\_REACCION{]}} = \text{{[}VENTAS\_CATEGORIAS{]}} \times \text{{[}PCT\_GAPS{]}} \times \frac{\text{{[}MEJORA\_MARGEN\_BPS{]}}}{10000} \times \text{{[}FACTOR\_VELOCIDAD{]}}
\end{equation}

\subsubsection*{Beneficio por Negociación con Marcas}
\begin{equation}
\text{{[}BENEFICIO\_NEGOCIACION{]}} = \text{{[}COMPRAS\_MARCAS\_CLAVE{]}} \times \frac{\text{{[}PCT\_MEJORA\_TERMINOS{]}}}{100}
\end{equation}

\subsubsection*{Cálculo de ROI}
\begin{equation}
\text{ROI (\%)} = \frac{\text{{[}SUMA\_BENEFICIOS{]}} - \text{{[}OPEX\_ANUAL{]}}}{\text{{[}CAPEX\_TOTAL{]}}} \times 100
\end{equation}

\subsubsection*{Break-even (meses)}
\begin{equation}
\text{Break-even} = \frac{\text{{[}CAPEX\_TOTAL{]}}}{\frac{\text{{[}SUMA\_BENEFICIOS{]}} - \text{{[}OPEX\_ANUAL{]}}}{12}}
\end{equation}

\subsection{Ejemplo de Cálculo (Valores Ilustrativos)}

\begin{tcolorbox}[colback=primaryblue!10, colframe=primaryblue, title=\textbf{Ejemplo No Vinculante}]
\textit{Nota: Estos valores son solo para ilustrar el método de cálculo. Deben reemplazarse con datos reales.}

\vspace{0.3cm}
\textbf{Supuestos ejemplo:}
\begin{itemize}[leftmargin=*]
    \item Categorías objetivo: 15\% de ventas totales (\$10M/mes)
    \item Gap detectable: 40\% del tiempo
    \item Mejora de margen: 50 bps al reaccionar
    \item Reacción 2x más rápida con plataforma propia
\end{itemize}

\textbf{Beneficio anual estimado:}
\[
= \$10M \times 15\% \times 40\% \times 0.5\% \times 2 = \$60K/año
\]

\textit{Este es solo un ejemplo. Los valores reales pueden ser mayores o menores según las condiciones específicas del negocio.}
\end{tcolorbox}

\subsection{Responsables de Completar Variables}

\begin{table}[h]
\centering
\begin{tabularx}{\textwidth}{|X|X|c|}
\hline
\rowcolor{lightgray}
\textbf{Área} & \textbf{Variables a Completar} & \textbf{Deadline} \\
\hline
IT / PMO & CAPEX, OPEX técnico, N\_FTE & Semana 1 \\
\hline
Finanzas & Costos FTE, proveedor actual, ventas & Semana 1 \\
\hline
Comercial / Pricing & Beneficios, márgenes, gaps & Semana 2 \\
\hline
Equipo conjunto & Validación de cálculo final ROI & Semana 2 \\
\hline
\end{tabularx}
\caption{Plan para completar análisis financiero}
\end{table}

\end{document}
