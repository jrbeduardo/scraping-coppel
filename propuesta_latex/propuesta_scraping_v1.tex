\documentclass[12pt,a4paper]{article}

% Paquetes
\usepackage[utf8]{inputenc}
\usepackage[spanish]{babel}
\usepackage{geometry}
\usepackage{graphicx}
\usepackage{xcolor}
\usepackage{hyperref}
\usepackage{enumitem}
\usepackage{tabularx}
\usepackage{booktabs}
\usepackage{longtable}
\usepackage{fancyhdr}
\usepackage{listings}
\usepackage{amsmath}
\usepackage{amssymb}
\usepackage{tikz}
\usepackage{textcomp}
\usepackage{newunicodechar}

% Definir caracteres Unicode personalizados
\newunicodechar{✓}{\textcolor{successgreen}{\checkmark}}
\newunicodechar{✅}{\textcolor{successgreen}{[\checkmark]}}
\newunicodechar{❌}{\textcolor{dangerred}{[X]}}
\newunicodechar{⚠}{\textcolor{warningyellow}{/!$\backslash$}}
\newunicodechar{📊}{[Graf]}
\newunicodechar{📉}{[Down]}
\newunicodechar{📈}{[Up]}
\newunicodechar{🎯}{[Target]}
\newunicodechar{🔧}{[Tool]}
\newunicodechar{💰}{[\$]}
\newunicodechar{🔍}{[Search]}
\newunicodechar{💡}{[Idea]}
\newunicodechar{🚨}{[Alert]}
\newunicodechar{◯}{$\circ$}
\newunicodechar{🚀}{[Go]}
\newunicodechar{🛠}{[Build]}
\newunicodechar{📞}{[Tel]}

% Configuración de página
\geometry{
    top=2.5cm,
    bottom=2.5cm,
    left=2.5cm,
    right=2.5cm,
    headheight=15pt
}

% Configuración de colores
\definecolor{primaryblue}{RGB}{0,102,204}
\definecolor{successgreen}{RGB}{40,167,69}
\definecolor{warningyellow}{RGB}{255,193,7}
\definecolor{dangerred}{RGB}{220,53,69}

% Configuración de hyperlinks
\hypersetup{
    colorlinks=true,
    linkcolor=primaryblue,
    urlcolor=primaryblue,
    citecolor=primaryblue
}

% Encabezado y pie de página
\pagestyle{fancy}
\fancyhf{}
\fancyhead[L]{\textit{Plataforma de Scraping - MVP}}
\fancyhead[R]{\textit{Versión 0.2}}
\fancyfoot[C]{\thepage}

% Título
\title{
    \LARGE\textbf{Plataforma Interna de Scraping y Análisis\\Competitivo de Precios (MVP)}\\
    \vspace{0.5cm}
    \large Propuesta Técnica y de Negocio
}

\author{
    \textbf{Patrocinador:} Luis Acosta / Dirección Estrategia\\
    \textbf{Equipo Responsable:} Ruíz / Lozas
}

\date{15 de octubre de 2025}

\begin{document}

\maketitle
\thispagestyle{empty}

\vspace{1cm}
\begin{center}
    \Large
    \textcolor{successgreen}{\textbf{✓ RECOMENDADO - GO}}
\end{center}

\newpage
\tableofcontents
\newpage

% ============================================
% RESUMEN EJECUTIVO
% ============================================
\section{Resumen Ejecutivo}

Construiremos una plataforma interna que extrae de forma programada información pública de competidores (precio, promoción, disponibilidad y vendedor), la normaliza y la disponibiliza en una interfaz simple y mediante descargas/API.

El MVP (3/4 meses) se acota a \textbf{3/4 competidores} y \textbf{2/3 categorías} con frecuencia diaria (o menor solo si es responsable y viable), y comparación por SKU con matching básico (exacto/variante). Sitios de alta fricción entran en Fase 2 con ventanas y frecuencia reducidas.

\subsection*{Beneficio Esperado}

Habilitar playbooks de precio y reacción táctica ante movimientos de mercado, ganar trazabilidad histórica para negociar con marcas, reducir dependencia de un proveedor que hoy no cumple capacidades/SLAs.

\subsection*{Evaluación de Viabilidad Técnica}

\textbf{Status:} \textcolor{successgreen}{ALTAMENTE VIABLE - Se recomienda proceder}

\subsubsection*{Análisis del Proveedor Actual}

\begin{itemize}[leftmargin=*]
    \item \textcolor{dangerred}{Cobertura deficiente:} 54\% de productos sin competidor identificado
    \item \textcolor{dangerred}{Matching limitado:} Solo coincidencias exactas, sin variantes
    \item \textcolor{successgreen}{Extracción rica:} $\sim$50 atributos por producto (ventaja temporal)
    \item \textcolor{dangerred}{Sin trazabilidad:} No hay evidencia de históricos robustos
\end{itemize}

\subsubsection*{Ventajas Competitivas del MVP}

\begin{itemize}[leftmargin=*]
    \item \textbf{Matching mejorado:} Target 70-85\% vs 46\% actual
    \item \textbf{Frescura garantizada:} Datos <24h en >90\% SKUs
    \item \textbf{Control total:} Pipeline propio, sin dependencias
    \item \textbf{ROI positivo:} Break-even en 2-4 meses
\end{itemize}

% ============================================
% ANTECEDENTES Y PROBLEMA
% ============================================
\section{Antecedentes y Problema}

\subsection{Situación Actual}

\begin{itemize}[leftmargin=*]
    \item Dependencia de un tercero con resultados irregulares en frescura, cobertura y precisión
    \item Decisiones comerciales con latencia y poca evidencia histórica
    \item Necesitamos un pipeline propio con alcance realista, riesgos controlados y KPIs claros
\end{itemize}

\subsection{Análisis de Datos del Proveedor Actual}

\subsubsection*{Archivo 1 - analyse\_item\_list (107 productos)}
\begin{itemize}[leftmargin=*]
    \item Extracción detallada con $\sim$50 atributos específicos por categoría
    \item Incluye: precio, descuento, marca, seller, disponibilidad, envío, planes EMI
    \item Atributos técnicos: capacidad, modelo, tecnología, certificaciones
\end{itemize}

\subsubsection*{Archivo 2 - exact\_match\_data (200 productos de electrónicos)}
\begin{itemize}[leftmargin=*]
    \item \textcolor{warningyellow}{46\% productos ``Out Of Stock''}
    \item \textcolor{dangerred}{54\% productos ``No Competitor''} $\leftarrow$ Principal problema
    \item Solo matching básico (campo ``Difference'' = 0 en todos los casos)
    \item Categoría: Refrigeradores (Samsung 41, LG 31, Mabe 25, Whirlpool 24)
    \item Rango de precios: \$4,599 - \$91,999 MXN
\end{itemize}

\textbf{Conclusión:} El proveedor actual \textcolor{dangerred}{NO} está cumpliendo con las expectativas de cobertura y matching.

% ============================================
% OBJETIVO DEL MVP
% ============================================
\section{Objetivo del MVP (3/4 meses)}

\begin{enumerate}[leftmargin=*]
    \item Disponibilizar snapshots programados (diarios; sub-diarios cuando sea viable)
    \item Comparar por SKU: precio, precio lista, \% descuento, envío (si es visible), disponibilidad y tipo de vendedor (1P/3P)
    \item UI simple (filtros, tabla comparativa, series de tiempo) + descargar (CSV/Parquet) + API interna + alertas por umbrales
    \item Matching v1 (exacto/variante) usando claves duras (EAN/UPC/MPN/SKU) y reglas simples (pack/talla/color)
\end{enumerate}

\subsection{KPIs Objetivo vs Proveedor Actual}

\begin{table}[h]
\centering
\begin{tabularx}{\textwidth}{|X|c|c|c|}
\hline
\textbf{Métrica} & \textbf{Proveedor Actual} & \textbf{Target MVP} & \textbf{Mejora} \\
\hline
Cobertura efectiva & $\sim$46\% & $\geq$70\% & +52\% \\
\hline
Frescura (<24h) & ¿Semanal? & $\geq$90\% & \textcolor{successgreen}{✓} \\
\hline
Precisión precio & ¿? & $\geq$97\% & \textcolor{successgreen}{✓} \\
\hline
Disponibilidad sistema & ¿? & $\geq$97\% & \textcolor{successgreen}{✓} \\
\hline
\end{tabularx}
\caption{Comparativa de métricas: Proveedor Actual vs MVP}
\end{table}

% ============================================
% ALCANCE DEL MVP
% ============================================
\section{Alcance del MVP}

\subsection{Incluye}

\subsubsection*{Funcionalidad Core}
\begin{itemize}[leftmargin=*]
    \item Extracción de listados de búsqueda y páginas de producto en 3/4 competidores y 2/3 categorías
    \item Normalización de moneda, marca, pack/talla, categoría estándar
    \item Matching v1: exacto/variante
    \item UI con filtros, tabla comparativa, serie de tiempo y detalle con evidencias (URL y mini-captura)
    \item API/exports y alertas (p. ej., caída/subida de precio >x\% o cambio de disponibilidad)
\end{itemize}

\subsubsection*{Atributos a Extraer (MVP)}
\begin{enumerate}[leftmargin=*]
    \item Precio actual
    \item Precio lista / precio tachado
    \item \% Descuento
    \item Disponibilidad (In Stock / Out of Stock)
    \item Vendedor (1P / 3P + nombre si aplica)
    \item URL del producto
    \item Marca
    \item Categoría
    \item SKU del competidor
\end{enumerate}

\subsection{No Incluye (Fase 2)}

\subsubsection*{Fuera de Alcance MVP}
\begin{itemize}[leftmargin=*]
    \item Similaridad avanzada (texto/imagen con ML)
    \item Share of search
    \item Ratings/reviews
    \item Cobertura masiva de todos los sitios y categorías
    \item Sitios con alta fricción (ej.: algunos marketplaces globales)
    \item \textbf{Atributos técnicos detallados} (capacidad, color, tecnología, etc.) $\leftarrow$ El proveedor actual los tiene
\end{itemize}

\subsection{Recomendación: Plan para Fase 2}

\textbf{Atributos técnicos (Semanas 13-20):}
\begin{itemize}[leftmargin=*]
    \item Implementar extracción de especificaciones por categoría
    \item Usar selectores CSS específicos + plantillas configurables
    \item Considerar LLM API (GPT-4o/Claude) para extracción de atributos no estructurados
    \item \textbf{Prioridad:} Solo si el negocio lo requiere para decisiones comerciales
\end{itemize}

% ============================================
% CASOS DE USO
% ============================================
\section{Casos de Uso Habilitados}

\subsection{Playbooks Operativos}
\begin{itemize}[leftmargin=*]
    \item \textbf{Playbooks de precio:} detectar gaps y definir respuesta (mantener/igualar/contraatacar)
    \item \textbf{Oportunidad por OOS competidor:} cuando el competidor queda sin stock
    \item \textbf{Negociación con marcas:} evidencia histórica de movimientos de precio/promoción
    \item \textbf{Alertas operativas:} eventos relevantes para equipos comerciales
\end{itemize}

\subsection{Casos de Uso Adicionales Identificados}

Basados en el análisis de datos:
\begin{itemize}[leftmargin=*]
    \item \textbf{Detección de cambios de seller:} Competidor cambia de 1P a 3P
    \item \textbf{Alertas de reposición:} Competidor recupera stock después de OOS
    \item \textbf{Análisis de planes de financiamiento:} Si competidor ofrece MSI o planes EMI más agresivos
    \item \textbf{Monitoreo de envío:} Cambios en costos/tiempos de envío
\end{itemize}

% ============================================
% ARQUITECTURA
% ============================================
\section{Arquitectura del Sistema}

\subsection{Componentes Principales}

La arquitectura propuesta se organiza en cinco capas principales:

\subsubsection*{1. Capa de Extracción}
\begin{itemize}[leftmargin=*]
    \item Playwright/Puppeteer para renderizado de JavaScript
    \item Pool de workers con límites por dominio
    \item Proxy rotatorio (Bright Data/Smartproxy)
    \item Detección de cambios (hash estructura HTML)
\end{itemize}

\subsubsection*{2. Orquestación}
\begin{itemize}[leftmargin=*]
    \item Prefect/Dagster/Airflow para planificación de tareas
    \item DAGs por competidor/categoría
    \item Retry logic inteligente
    \item Rate limiting por dominio
\end{itemize}

\subsubsection*{3. Procesamiento y Matching}
\begin{itemize}[leftmargin=*]
    \item Normalización (precios, moneda, unidades)
    \item Matching v1: exacto (EAN/UPC) + variante
    \item Fuzzy matching (fuzzywuzzy/rapidfuzz)
    \item Validaciones (precios, outliers)
\end{itemize}

\subsubsection*{4. Almacenamiento}
\begin{itemize}[leftmargin=*]
    \item Raw: S3/GCS + Parquet (fecha/competidor)
    \item Procesado: DuckDB/ClickHouse
    \item Histórico: retención 12-24 meses
\end{itemize}

\subsubsection*{5. Capa de Publicación}
\begin{itemize}[leftmargin=*]
    \item API: FastAPI (endpoints REST)
    \item UI: Streamlit/Retool
    \item Alertas: Email/Slack
\end{itemize}

\subsection{Recomendaciones Tecnológicas}

\begin{table}[h]
\centering
\small
\begin{tabularx}{\textwidth}{|X|c|X|}
\hline
\textbf{Componente} & \textbf{Recomendado} & \textbf{Justificación} \\
\hline
Scraping & Playwright & Mejor manejo de SPA, APIs más limpias \\
\hline
Orquestación & Prefect & Más moderno, mejor DX, Python-native \\
\hline
Storage (raw) & S3 + Parquet & Costo, integración \\
\hline
Storage (queries) & DuckDB & OLAP sobre Parquet, zero-config \\
\hline
API & FastAPI & Performance, docs automáticas, async \\
\hline
UI & Streamlit & Prototipado rápido, Python-only \\
\hline
Proxies & Bright Data & Mejor uptime, más IPs residenciales \\
\hline
Matching & rapidfuzz & Más rápido que fuzzywuzzy \\
\hline
Monitoring & Grafana + Loki & Open-source, menor costo \\
\hline
\end{tabularx}
\caption{Stack tecnológico recomendado}
\end{table}

% ============================================
% MODELO DE DATOS
% ============================================
\section{Modelo de Datos}

\subsection{Dimensiones (Catálogos)}

\subsubsection*{Producto Canónico}
\begin{lstlisting}[language=SQL, basicstyle=\small\ttfamily, frame=single]
dim_producto:
  - producto_id (PK)
  - ean / upc / mpn
  - marca
  - familia
  - clase
  - departamento
  - pack / talla / color
  - categoria_estandar
\end{lstlisting}

\subsubsection*{Competidor}
\begin{lstlisting}[language=SQL, basicstyle=\small\ttfamily, frame=single]
dim_competidor:
  - competidor_id (PK)
  - nombre
  - dominio
  - tipo (retailer/marketplace)
  - limite_rpm (rate limit)
\end{lstlisting}

\subsubsection*{Vendedor}
\begin{lstlisting}[language=SQL, basicstyle=\small\ttfamily, frame=single]
dim_vendedor:
  - vendedor_id (PK)
  - nombre
  - tipo (1P/3P)
  - competidor_id (FK)
\end{lstlisting}

\subsection{Hechos (Mediciones)}

\subsubsection*{Precio Histórico}
\begin{lstlisting}[language=SQL, basicstyle=\small\ttfamily, frame=single]
fact_precio:
  - precio_id (PK)
  - producto_id (FK)
  - competidor_id (FK)
  - vendedor_id (FK)
  - fecha_id (FK)
  - precio_actual
  - precio_lista
  - descuento_monto
  - descuento_porcentaje
  - disponibilidad (boolean)
  - moneda
  - url
  - timestamp_extraccion
  - hash_evidencia (mini-captura)
\end{lstlisting}

\subsection{Tabla de Matching}

\begin{lstlisting}[language=SQL, basicstyle=\small\ttfamily, frame=single]
rel_matching:
  - matching_id (PK)
  - producto_canonico_id (FK)
  - competidor_id (FK)
  - sku_competidor
  - tipo_match (exacto/variante/fuzzy)
  - score_similitud (0-100)
  - fecha_validacion
  - validado_manualmente (boolean)
\end{lstlisting}

\textbf{Tipos de match:}
\begin{itemize}[leftmargin=*]
    \item \textbf{Exacto:} EAN/UPC/MPN coincide 100\%
    \item \textbf{Variante:} Mismo producto, diferente talla/color/pack
    \item \textbf{Fuzzy:} Similitud >85\% en nombre normalizado (Fase MVP tardía)
\end{itemize}

% ============================================
% KPIs Y CRITERIOS DE ÉXITO
% ============================================
\section{KPIs y Criterios de Éxito (SLOs)}

\subsection{Métricas Core del MVP}

\begin{longtable}{|p{3cm}|p{3cm}|p{7cm}|}
\hline
\textbf{KPI} & \textbf{Target MVP} & \textbf{Método de medición} \\
\hline
\endfirsthead
\hline
\textbf{KPI} & \textbf{Target MVP} & \textbf{Método de medición} \\
\hline
\endhead
Cobertura & $\geq$70\% de SKUs prioritarios con $\geq$1 coincidencia & \texttt{COUNT(DISTINCT matched\_skus) / COUNT(golden\_set)} \\
\hline
Frescura & $\geq$90\% de SKUs con datos de las últimas 24h & \texttt{COUNT(WHERE timestamp > NOW()-24h) / total\_skus} \\
\hline
Precisión (precio/promo) & $\geq$97\% en muestreo estratificado & Validación manual semanal (50 SKUs) \\
\hline
Disponibilidad de procesos & $\geq$97\% & Uptime de procesos de extracción \\
\hline
Tiempo de recuperación ante cambio de página & <24h & MTTR desde detección hasta fix \\
\hline
\caption{Métricas y objetivos del MVP}
\end{longtable}

\subsection{Definición de Éxito del MVP}

Cumplir estos indicadores en \textbf{$\geq$2 competidores} y \textbf{$\geq$2 categorías} durante \textbf{4 semanas continuas}.

\subsection{Comparativa con Proveedor Actual}

\begin{table}[h]
\centering
\begin{tabularx}{\textwidth}{|X|c|c|c|}
\hline
\textbf{Aspecto} & \textbf{Proveedor Actual} & \textbf{Target MVP} & \textbf{Mejora} \\
\hline
Matching efectivo & $\sim$46\% & $\geq$70\% & \textcolor{successgreen}{\textbf{+52\%}} \\
\hline
Frescura & Semanal (?) & Diaria (>90\%) & \textcolor{successgreen}{\textbf{✓}} \\
\hline
Precisión & Desconocida & $\geq$97\% & \textcolor{successgreen}{\textbf{✓}} \\
\hline
MTTR cambios & Desconocido & <24h & \textcolor{successgreen}{\textbf{✓}} \\
\hline
Trazabilidad & Limitada & Completa & \textcolor{successgreen}{\textbf{✓}} \\
\hline
\end{tabularx}
\caption{Comparativa de mejoras respecto al proveedor actual}
\end{table}

% ============================================
% PLAN DE TRABAJO
% ============================================
\section{Plan de Trabajo (12 semanas)}

\subsection{Semanas 0-2: Descubrimiento y Base}

\textbf{Objetivos:}
\begin{itemize}[leftmargin=*]
    \item Validar competidores y categorías; levantar golden set (500/1,000 SKUs)
    \item Diseñar modelo de datos y mockups de UI; revisar términos y robots por dominio
    \item Definir KPIs y ``Definition of Done''
\end{itemize}

\textbf{Entregables:}
\begin{itemize}[leftmargin=*]
    \item[$\Box$] Lista de 3-4 competidores aprobada
    \item[$\Box$] Lista de 2-3 categorías con palabras de búsqueda
    \item[$\Box$] Golden set con EAN/UPC/MPN cuando exista
    \item[$\Box$] Mockups de UI
    \item[$\Box$] Matriz legal (ToS y robots.txt por dominio)
\end{itemize}

\textbf{\textcolor{successgreen}{Quick win:}} Presentación de mockups a usuarios finales

\subsection{Semanas 3-4: Infraestructura y Primer Sitio}

\textbf{Objetivos:}
\begin{itemize}[leftmargin=*]
    \item Configurar repos, planificador, bitácoras y almacenamiento
    \item Implementar primer sitio (búsquedas + producto) y normalización
    \item UI v0 (tabla + filtros) y descarga básica
\end{itemize}

\textbf{Entregables:}
\begin{itemize}[leftmargin=*]
    \item[$\Box$] Repo configurado + CI/CD básico
    \item[$\Box$] Primer scraper funcional (1 competidor, 1 categoría)
    \item[$\Box$] Storage en Parquet funcionando
    \item[$\Box$] UI v0 con tabla comparativa básica
\end{itemize}

\textbf{\textcolor{primaryblue}{Hito crítico (Semana 4):}} Demo funcional con datos reales de 1 competidor

\subsection{Semanas 5-6: Más Sitios y Matching}

\textbf{Objetivos:}
\begin{itemize}[leftmargin=*]
    \item Implementar sitio 2 y sitio 3; control de calidad por muestreo
    \item Matching v1 (exacto/variante); API y UI v1 (serie de tiempo)
\end{itemize}

\textbf{Entregables:}
\begin{itemize}[leftmargin=*]
    \item[$\Box$] 3 competidores scraped diariamente
    \item[$\Box$] Matching exacto por EAN/UPC
    \item[$\Box$] Matching de variantes (talla/color/pack)
    \item[$\Box$] API endpoints básicos
    \item[$\Box$] UI con serie de tiempo
\end{itemize}

\textbf{\textcolor{successgreen}{KPI checkpoint:}} Cobertura >50\% en golden set

\subsection{Semanas 7-8: Robustez y Alertas}

\textbf{Objetivos:}
\begin{itemize}[leftmargin=*]
    \item Detector de cambios de página; reintentos inteligentes; alertas por umbrales
    \item Demostración con usuarios comerciales y ajustes
\end{itemize}

\textbf{Entregables:}
\begin{itemize}[leftmargin=*]
    \item[$\Box$] Detector de cambios de estructura
    \item[$\Box$] Sistema de alertas configurables
    \item[$\Box$] Retry logic inteligente
    \item[$\Box$] Demo con usuarios comerciales
    \item[$\Box$] Feedback documentado
\end{itemize}

\textbf{\textcolor{successgreen}{Validación de usuarios:}} 5 usuarios clave validan la plataforma

\subsection{Semanas 9-12: Ampliación y Cierre del MVP}

\textbf{Objetivos:}
\begin{itemize}[leftmargin=*]
    \item Sumar categoría \#2 (y \#3 si aplica); pruebas de carga
    \item Monitoreo de KPIs 4 semanas; decisión Go/No-Go y backlog de Fase 2
\end{itemize}

\textbf{Entregables:}
\begin{itemize}[leftmargin=*]
    \item[$\Box$] 2-3 categorías completamente operativas
    \item[$\Box$] 4 semanas continuas cumpliendo KPIs
    \item[$\Box$] Documentación completa
    \item[$\Box$] Plan de Fase 2
    \item[$\Box$] Decisión Go/No-Go para escalamiento
\end{itemize}

\textbf{\textcolor{primaryblue}{Decisión final:}} ¿Proceder con Fase 2 o ajustar?

% ============================================
% EQUIPO Y ROLES
% ============================================
\section{Equipo Mínimo y Roles}

\begin{longtable}{|p{4cm}|c|p{7cm}|}
\hline
\textbf{Rol} & \textbf{Dedicación} & \textbf{Responsabilidades} \\
\hline
\endfirsthead
\hline
\textbf{Rol} & \textbf{Dedicación} & \textbf{Responsabilidades} \\
\hline
\endhead
Líder técnico / Datos Sr & 1.0 FTE & Arquitectura, orquestación, observabilidad, robustez \\
\hline
Ingeniero/a de datos & 1.0 FTE & Extracción, normalización, procesos, matching \\
\hline
Ingeniero/a back/frontend & 1.0 FTE & API, autenticación, interfaz, descargas, alertas \\
\hline
PM/PO & 0.75 FTE & Roadmap, riesgos, relación con usuarios y sponsors \\
\hline
\textbf{TOTAL} & \textbf{3.75 FTE} & \textbf{Equipo MVP} \\
\hline
\caption{Equipo y dedicación requerida}
\end{longtable}

\textbf{Fase 2 (opcional):} +0.5–1.0 analista/ML para similaridad avanzada

% ============================================
% RIESGOS Y MITIGACIONES
% ============================================
\section{Riesgos y Mitigaciones}

\begin{longtable}{|c|p{3.5cm}|c|c|p{5cm}|}
\hline
\textbf{\#} & \textbf{Riesgo} & \textbf{Prob.} & \textbf{Impacto} & \textbf{Mitigación} \\
\hline
\endfirsthead
\hline
\textbf{\#} & \textbf{Riesgo} & \textbf{Prob.} & \textbf{Impacto} & \textbf{Mitigación} \\
\hline
\endhead
1 & Defensas anti-extracción en sitios & Alta & Alto & Proxies residenciales + cadencia conservadora + headers realistas \\
\hline
2 & Cambios en estructura de páginas & Media & Alto & Detector de cambios automático. MTTR <24h \\
\hline
3 & Calidad de matching baja & Media & Medio & Golden set robusto (500-1K SKUs). Fuzzy matching \\
\hline
4 & Cumplimiento legal/operativo & Baja & Alto & Matriz por dominio (ToS y robots.txt). Solo info pública \\
\hline
5 & Costo de conectividad/IPs & Media & Medio & Medición por dominio. Budget cap mensual \\
\hline
6 & Scope creep (features adicionales) & Alta & Medio & \textbf{Stick to MVP.} Decir NO a features \\
\hline
7 & Cambios frecuentes en marketplaces & Alta & Alto & Empezar con retailers directos \\
\hline
8 & Equipo no completo a tiempo & Media & Alto & Pre-asignar equipo antes de Sprint 0 \\
\hline
\caption{Matriz de riesgos y estrategias de mitigación}
\end{longtable}

% ============================================
% COSTOS
% ============================================
\section{Costos y ROI}

\subsection{Inversión Inicial (CAPEX)}

\begin{itemize}[leftmargin=*]
    \item Desarrollo del MVP: 3.75 FTE $\times$ 3 meses $\times$ \$10,000 USD promedio
    \item \textbf{Total CAPEX: $\sim$\$112,500 USD}
\end{itemize}

\subsection{Costos Operativos (OPEX Mensual)}

\begin{table}[h]
\centering
\begin{tabularx}{\textwidth}{|X|c|c|c|}
\hline
\textbf{Concepto} & \textbf{Conservador} & \textbf{Base} & \textbf{Ambicioso} \\
\hline
Cómputo/workers & \$150 & \$300 & \$500 \\
\hline
Almacenamiento (S3/GCS) & \$50 & \$100 & \$200 \\
\hline
Proxies/IPs rotatorias & \$300 & \$800 & \$1,500 \\
\hline
Monitoreo (logs/métricas) & \$50 & \$100 & \$150 \\
\hline
Orquestación (Prefect Cloud) & \$0 & \$0 & \$200 \\
\hline
Contingencia (10\%) & \$55 & \$130 & \$255 \\
\hline
\textbf{TOTAL MENSUAL} & \textbf{\$605} & \textbf{\$1,430} & \textbf{\$2,805} \\
\hline
\textbf{TOTAL ANUAL} & \textbf{\$7,260} & \textbf{\$17,160} & \textbf{\$33,660} \\
\hline
\end{tabularx}
\caption{Costos operativos en diferentes escenarios}
\end{table}

\subsection{Comparativa Financiera}

\begin{table}[h]
\centering
\begin{tabularx}{\textwidth}{|X|c|c|c|}
\hline
\textbf{Concepto} & \textbf{Proveedor Actual} & \textbf{MVP Año 1} & \textbf{MVP Año 2+} \\
\hline
Costo total & \$36,000-96,000 & \$129,660 & \$17,160 \\
\hline
Control & \textcolor{dangerred}{Nulo} & \textcolor{successgreen}{Total} & \textcolor{successgreen}{Total} \\
\hline
Cobertura & $\sim$46\% & 70-85\% & 70-85\%+ \\
\hline
\end{tabularx}
\caption{Análisis financiero comparativo}
\end{table}

\textbf{Break-even:} Si el proveedor cobra >\$5,000/mes $\rightarrow$ \textcolor{successgreen}{\textbf{ROI positivo en 2-4 meses}}

\textbf{Ahorro anual (desde Año 2):} \$36,000-96,000 - \$17,160 = \textcolor{successgreen}{\textbf{\$18,840-78,840 USD/año}}

% ============================================
% GOBIERNO Y SEGURIDAD
% ============================================
\section{Gobierno, Auditoría y Seguridad}

\subsection{Gobierno del Proyecto}

\textbf{Comité quincenal:}
\begin{itemize}[leftmargin=*]
    \item Sponsor (Luis Acosta)
    \item PM/PO
    \item Líder técnico
    \item Representante usuarios comerciales
    \item Representante legal (Q\&A sobre compliance)
\end{itemize}

\textbf{Agenda estándar:}
\begin{itemize}[leftmargin=*]
    \item Avance vs plan (semáforo)
    \item KPIs actuales
    \item Riesgos top 3
    \item Decisiones requeridas
    \item Budget burn rate
\end{itemize}

\subsection{Seguridad}

\subsubsection*{Acceso}
\begin{itemize}[leftmargin=*]
    \item SSO corporativo (Azure AD / Okta)
    \item RBAC (roles: admin, comercial, analista, auditor)
    \item MFA obligatorio para admins
\end{itemize}

\subsubsection*{Datos}
\begin{itemize}[leftmargin=*]
    \item Cifrado en tránsito (TLS 1.3)
    \item Cifrado en reposo (S3 server-side encryption)
    \item Sin PII de clientes
    \item Logs de acceso a datos sensibles
\end{itemize}

\subsubsection*{Compliance}
\begin{itemize}[leftmargin=*]
    \item Revisión legal de ToS por dominio (cada 6 meses)
    \item Respeto de robots.txt
    \item Rate limiting documentado
    \item Proceso de opt-out si un competidor lo solicita
\end{itemize}

% ============================================
% DECISIONES SOLICITADAS
% ============================================
\section{Decisiones Solicitadas al Sponsor}

\subsection{Decisiones Críticas (Esta Semana)}

\begin{enumerate}[leftmargin=*]
    \item \textcolor{successgreen}{✓} \textbf{Aprobación del alcance del MVP}
    \item \textcolor{dangerred}{◯} \textbf{Definición de competidores y categorías} (deadline: esta semana)
    \begin{itemize}
        \item Propuesta: Liverpool, Elektra, Palacio de Hierro
        \item Categorías: Línea Blanca, Electrónicos
    \end{itemize}
    \item \textcolor{warningyellow}{◯} \textbf{Inicio del Sprint 0} (2 semanas): descubrimiento, legal y base técnica
    \item \textcolor{warningyellow}{◯} \textbf{Primer hito visible} (Semana 4): tabla comparativa funcional con 1 sitio
\end{enumerate}

\subsection{Decisiones Secundarias (2 Semanas)}

\begin{enumerate}[leftmargin=*, start=5]
    \item Plan de comunicación a equipos comerciales
    \item Proceso de feedback de usuarios durante MVP
    \item Criterios de éxito para aprobar Fase 2
\end{enumerate}

% ============================================
% PRÓXIMOS PASOS
% ============================================
\section{Próximos Pasos Inmediatos}

\subsection{Esta Semana}
\begin{enumerate}[leftmargin=*]
    \item \textbf{Sponsor aprueba documento y presupuesto}
    \item \textbf{Definir competidores finales} (Liverpool, Elektra, Palacio + 1?)
    \item \textbf{Definir categorías finales} (Línea Blanca, Electrónicos + 1?)
    \item \textbf{Pre-asignar equipo} (4 personas con nombres)
\end{enumerate}

\subsection{Semana Próxima (Semana 0 - Inicio)}
\begin{enumerate}[leftmargin=*, start=5]
    \item \textbf{Kickoff del proyecto} con equipo completo
    \item \textbf{Construir golden set} (500 SKUs prioritarios)
    \item \textbf{Revisar ToS y robots.txt} de competidores
    \item \textbf{Setup técnico inicial} (repo, cloud, proxies trial)
\end{enumerate}

\subsection{Semana 4 (Primer Hito)}
\begin{enumerate}[leftmargin=*, start=9]
    \item \textbf{Demo funcional} con datos reales de 1 competidor
    \item \textbf{Validación con usuarios} (5 personas comerciales)
\end{enumerate}

% ============================================
% GLOSARIO
% ============================================
\section{Glosario}

\begin{description}[leftmargin=!, labelwidth=4cm]
    \item[Extracción (scraping)] Obtención automatizada de información pública mostrada en páginas web
    \item[Snapshot] Captura de estado (precio/stock) en una fecha/hora específica
    \item[SKU canónico] Identificador interno para comparar ``manzanas con manzanas'' entre competidores
    \item[Matching exacto] Mismo producto identificado por EAN/UPC/MPN
    \item[Matching variante] Misma referencia con diferencias (talla, color, pack)
    \item[Matching fuzzy] Similitud aproximada basada en texto (>85\% similitud)
    \item[1P (First Party)] Producto vendido directamente por el retailer
    \item[3P (Third Party)] Producto vendido por un seller externo en marketplace
    \item[Golden set] Conjunto de SKUs prioritarios para QA y validación
    \item[MTTR] Mean Time To Recovery - tiempo promedio de recuperación ante fallas
    \item[SLO] Service Level Objective - objetivo de nivel de servicio
\end{description}

% ============================================
% RECOMENDACIÓN FINAL
% ============================================
\section*{Recomendación Final}

\begin{center}
\fbox{\begin{minipage}{0.9\textwidth}
\vspace{0.5cm}
\begin{center}
\LARGE\textcolor{successgreen}{\textbf{✓ RECOMENDACIÓN: GO}}
\end{center}
\vspace{0.5cm}

Este MVP es técnicamente viable, financieramente justificable, y estratégicamente necesario dado el desempeño deficiente del proveedor actual (54\% sin matching).

\vspace{0.3cm}
\textbf{Confianza en éxito:} Alta (80\%)

\textbf{Riesgo principal:} Scope creep y anti-bot en sitios complejos

\textbf{Mitigación clave:} Disciplina en MVP, empezar con retailers simples

\vspace{0.3cm}
\begin{center}
\Large\textbf{¡Es momento de construir!}
\end{center}
\vspace{0.5cm}
\end{minipage}}
\end{center}

% ============================================
% FIRMA DE APROBACIÓN
% ============================================
\newpage
\section*{Firma de Aprobación}

\vspace{2cm}

\begin{tabularx}{\textwidth}{|l|X|c|c|}
\hline
\textbf{Rol} & \textbf{Nombre} & \textbf{Aprobación} & \textbf{Fecha} \\
\hline
Sponsor & Luis Acosta & $\Box$ Apruebo & \_\_/\_\_/\_\_ \\
\hline
Tech Lead & Lozas & $\Box$ Apruebo & \_\_/\_\_/\_\_ \\
\hline
PM & Ruíz & $\Box$ Apruebo & \_\_/\_\_/\_\_ \\
\hline
Legal & [Nombre] & $\Box$ Apruebo & \_\_/\_\_/\_\_ \\
\hline
\end{tabularx}

\vspace{2cm}

\begin{center}
\textit{Versión 0.2 - 15 de octubre de 2025}\\
\textit{Próxima revisión: Semana 4 (hito de demo funcional)}
\end{center}

\end{document}
